\documentclass[9pt,letterpaper]{article}
\usepackage[margin=0.7in]{geometry}
\usepackage{booktabs}
\usepackage{graphicx}
\usepackage{hyperref}
\usepackage{array}
\usepackage{enumitem}
\usepackage{xcolor}
\usepackage{fancyhdr}
\usepackage{multicol}
\usepackage{wrapfig}

% Tighter spacing
\setlength{\parskip}{1pt}
\setlength{\parindent}{0pt}
\setlength{\abovecaptionskip}{1pt}
\setlength{\belowcaptionskip}{1pt}
\setlist{nosep,leftmargin=12pt}
\setlength{\intextsep}{4pt}

% Header/Footer
\pagestyle{fancy}
\fancyhf{}
\rfoot{\small Page \thepage}
\renewcommand{\headrulewidth}{0pt}

% Custom colors
\definecolor{mitred}{RGB}{163,31,52}
\hypersetup{
    colorlinks=true,
    linkcolor=mitred,
    urlcolor=blue
}

\begin{document}

% Memo Header
\noindent
\textbf{To:} MIT Technology Policy Hackathon Judges \& KOSA Coalition Partners \\
\textbf{From:} MIT Hackathon Policy Team \\
\textbf{Date:} November 22, 2025 \\
\textbf{Re:} \textbf{Youth Online Safety: A Data-Driven Federal Framework to End Geographic Inequity}

\vspace{0.15cm}
\noindent\rule{\textwidth}{0.4pt}
\vspace{0.15cm}

% Executive Summary
\section*{Executive Summary}
\vspace{-0.1cm}

Analysis of \textbf{7,938 bills} (2020--2025) reveals \textbf{52\% of children live in low-protection states} (0--3 provisions) while only 19\% benefit from comprehensive safeguards (6--8 provisions). We recommend a \textbf{3-tier federal framework}: uniform standards where states show consensus (81\% platform liability, 64\% education, 58\% age verification), flexibility in moderate areas (23--64\% adoption), and state control for experimentation ($<$25\% adoption). This transforms the 1:278 federal-to-state ratio into coordinated protection while closing the 95\% evidence gap.

\vspace{-0.1cm}
\section*{Context}
\vspace{-0.1cm}

NLP analysis of 6,239 state bills reveals \textbf{states demonstrate remarkable agreement}: 81\% adopted platform liability, 64\% education, 58\% age verification. Yet Congress passed only 1 bill while states passed 278 (ratio: 1:278), creating regulatory chaos with 48 compliance regimes averaging 3.69 requirements/state. The 671 bills introduced in 2025 signal urgent momentum. The question isn't \textit{whether} to act, but \textit{how} to transform fragmented consensus into coordinated national protection.

\vspace{-0.1cm}
\section*{Recommendations}
\vspace{-0.1cm}

\subsection*{Tier 1: Uniform Federal Standards (High Consensus: 50--81\% State Adoption)}
\vspace{-0.05cm}

\textbf{Establish national minimums where states already agree to eliminate geographic inequity:}

\begin{enumerate}
    \item \textbf{Platform Duty of Care} (81\%): Codify obligation to prevent foreseeable harms through design, features, and algorithms. Hold platforms liable for negligent violations.
    \item \textbf{Age Verification} (58\%): Mandate privacy-preserving verification (zero-knowledge proofs, third-party services). Prohibit ID retention. Establish safe harbor.
    \item \textbf{Data Privacy} (50\%): Ban sale of minors' data. Require opt-in consent beyond core services. Mandate data minimization and annual audits.
    \item \textbf{Transparency \& Reporting} (38\%): Quarterly reports on minor users, moderation, incidents, algorithm effects. Addresses \textbf{95\% evidence gap}.
\end{enumerate}

\vspace{-0.05cm}
\subsection*{Tier 2: Federal Framework with State Flexibility (Moderate Consensus: 23--64\%)}
\vspace{-0.05cm}

\textbf{Set federal goals but allow state implementation to respect local values:}

\begin{enumerate}
    \item \textbf{Digital Literacy} (64\%): \$500M federal funding over 5 years. States design curricula aligned with local standards.
    \item \textbf{Parental Consent} (42\%): Federal defines ``meaningful consent.'' States choose implementation methods.
    \item \textbf{Content Moderation} (23\%): Federal establishes harm categories. States add regional priorities respecting First Amendment.
\end{enumerate}

\vspace{-0.05cm}
\subsection*{Tier 3: State Control (Low Consensus: $<$25\% Adoption)}
\vspace{-0.05cm}

\textbf{Preserve state experimentation:} School tech policies (19\%), time limits (7\%), design standards (2\%), state-specific enforcement, and flexible timelines.

\vspace{-0.1cm}
\section*{Analysis}
\vspace{-0.1cm}

\subsection*{Quantifying Geographic Inequity}
\vspace{-0.05cm}

\begin{wrapfigure}{r}{0.38\textwidth}
\vspace{-0.5cm}
\centering
\includegraphics[width=0.36\textwidth]{visualizations/geographic_inequity_analysis.png}
\caption{\footnotesize Geographic Inequity: 52\% low protection.}
\vspace{-0.4cm}
\end{wrapfigure}

\textbf{Geographic Inequity Index (GII)} = 2.06 (0--5 scale). Distribution: 52\% low-protection (0--3 provisions), 29\% medium (4--5), 19\% high (6--8). \textbf{Real impact:} California teens get 8 protections; Wyoming teens get zero.

\vspace{-0.05cm}
\subsection*{Hidden Consensus in State Legislation}
\vspace{-0.05cm}

\textbf{NLP analysis} identified 10 provisions. Top adoption: Platform Liability 81\%, Enforcement 71\%, Education 64\%, Age Verification 58\%, Data Privacy 50\%. \textbf{4 of 5 states agree} on platform accountability---consensus exists, federal leadership is absent.

\vspace{-0.05cm}
\subsection*{Federal Inaction Crisis}
\vspace{-0.05cm}

\begin{wrapfigure}{r}{0.38\textwidth}
\vspace{-0.5cm}
\centering
\includegraphics[width=0.36\textwidth]{visualizations/legislative_momentum_analysis.png}
\caption{\footnotesize 671 bills in 2025: urgent momentum.}
\vspace{-0.4cm}
\end{wrapfigure}

Congress: \textbf{1 bill}. States: \textbf{278 bills} (ratio 1:278). Creates: (1) \textbf{Compliance chaos}---48 regimes (avg. 3.69 requirements/state); (2) \textbf{Regulatory arbitrage}---platforms optimize for weakest states; (3) \textbf{Commerce disruption}---TikTok differs by state; (4) \textbf{Uncertainty}---startups face \$2--5M patchwork costs vs. \$500K--1M single standard.

\vspace{-0.05cm}
\subsection*{The 95\% Evidence Gap}
\vspace{-0.05cm}

500 sampled bills: 95.7\% lack privacy data, 95.0\% lack impact data, 100\% lack cost data. \textbf{Policymakers legislate blind.} Tier 1 transparency mandates generate evidence. 5-year sunset forces data-driven review.

\vspace{-0.05cm}
\subsection*{Methodology}
\vspace{-0.05cm}

\textbf{Data:} 7,938 bills (Integrity Institute). \textbf{NLP:} spaCy/NLTK, TF-IDF clustering. \textbf{Stats:} GII = coefficient of variation. \textbf{Validation:} Manual review. \href{https://github.com/Wv-Anterola/MIT-Policy-Hackathon}{Full code/data available}.

\vspace{-0.1cm}
\section*{Alternatives Considered}
\vspace{-0.1cm}

\textbf{Status Quo (Rejected):} Perpetuates 52\% low-protection crisis. Data crosses borders. Patchwork costs \$2--5M vs. \$500K--1M uniform. \textbf{Federal Preemption (Rejected):} Politically infeasible. Kills beneficial experimentation. Ignores regional differences. \textbf{Self-Regulation (Rejected):} Decade of failures. Perverse incentives conflict with safety. \textbf{Our Framework Wins:} (1) Data-driven tiers; (2) Respects federalism; (3) Politically feasible (81\% consensus); (4) Future-proof (sunset, evidence); (5) Equitable for all children.

\vspace{-0.1cm}
\section*{Implementation \& Trade-Offs}
\vspace{-0.1cm}

\textbf{Phased Rollout:} Y1: Transparency. Y2: Liability/privacy. Y3: Verification/education. Y3+: Small platforms ($<$1M users).

\textbf{Trade-Offs:} \textit{First Amendment}---targets design not content (\textit{Ginsberg} precedent). \textit{Privacy}---mandate zero-knowledge proofs, prohibit ID retention. \textit{Innovation}---standardization cheaper than chaos; safe harbors protect startups. \textit{Enforcement}---FTC leads Tier 1; states handle Tiers 2--3; private action for violations.

\vspace{-0.1cm}
\section*{Conclusion}
\vspace{-0.1cm}

\textbf{We have consensus} (81\% liability, 64\% education), \textbf{momentum} (671 bills in 2025), and \textbf{a solution} (3-tier framework). Missing: political will. Transform fragmented state action into federal leadership ensuring every child---California or Wyoming---enjoys equal protection. The alternative: continued inequity where zip code determines digital safety. \textbf{The time to act is now.}

\end{document}
